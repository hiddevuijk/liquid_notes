
In this section I will explain how to solve the PY + OZ 
in Fourier space.
This is faster than the real space method.

I will first show the naive implimentation, and point out some of the problems with it, then solve the problems and
show the final algorithm at the end of this section.
The naive algorithm is:
\begin{enumerate}
\item Guess an acceptable initial $h(r)$ denoted by $h_0(r)$.
 For example, use the low-density solution $h_0(r) = \lim_{\rho \to 0} h(r)$ or use an  $h(r)$ calculated for slightly different parameters.
\item Get the direct-correlation function $c_n(r)$ corresponding to $h_n(r)$ from the Percus-Yevick equation:
\begin{align}
c_n(r) = f(r)e^{\beta u(r)} \left( h_n(r) - 1\right).
\end{align}
For the first loop through the algorithm use $n=0$.
\item Fourier Transform $c_n(r)$ to get $C_n(k)$.
\item Use the Ornstein-Zernike equation to the next Fourier transformed total-correlation function:
\begin{align}
H_{n+1}(k) = \frac{C(k)}{1-\rho C(k)}.
\end{align}
\item Fourier transform back to real space to get $h'_{n+1}(r)$.
\item Mix the new solution with the old one:
\begin{align}
h_{n+1}(r) = \alpha h_n(r) + (1-\alpha) h'_{n+1}(r),
\end{align}
with $0\leq \alpha < 1$.
\item Check for convergence. If converged, you are done, else go to 2.
\end{enumerate}

This algorithm will lead to multiplication and divisions by zeros and infinities.
To avoid these, some alterations need to be made.

\subsubsection{The Indirect Correlation Function}
The Percus-Yevick Equation is 
\begin{align}
c(r) &= f(r)y(r)\\
&= \left[ e^{-\beta u(r)} - 1\right]
    e^{\beta u(r)} \left[ h(r) + 1\right] \\
&= h(r) - \left[ e^{\beta u(r)} h(r) +
    e^{\beta u(r)} - 1 \right],
\label{eq:py_problem}
\end{align}
where I used $f(r) = e^{-\beta u(r)}$, $y(r) = e^{\beta u(r)} g(r)$ and $h(r) = g(r) - 1$ to go from the first to the second line.
Both $c(r)$ and $h(r)$ are finite; therefor, the term in the 
square brackets is also finite.
The exponentials in the square brackets diverge for potentials that are strongly repulsive at short distances, which is the case for the Lennard-Jones or Hard sphere potential.
So there is some cancellation of infinities.
This is not a problem for theoretical manipulation,
but it makes numerical calculations impossible.
Because the expression in the square brackets is finite, the problem is only in the representation of those terms.
The problem can be solved by defining the indirect correlation function, $d(r)$
\footnote{This is not a commonly used function and the symbol for this function is not used in other texts.}
, and work with the direct and indirect-correlation functions instead of the direct and the total-corelation functions. 

The total-correlation function is the sum of the direct and the indirect-correlation function:
\begin{align}
h(r) = c(r) + d(r).
\end{align}
The Percus-Yevick approximation, is an approximate relation between the direct and total-correlation function, but it can be rewriten, such that it is an approximate relation between the direct and indirect-correlation function:
\begin{align}
c(r) = c(r) + d(r) - 
    \left[
    e^{\beta u(r)} \left( c(r) + d(r) \right) 
    + e^{\beta u(r)} - 1
    \right].
\end{align}
Reorder this to get:
\begin{align}
e^{\beta u(r)} c(r) &= d(r) - e^{\beta u(r)} +
        \left( 1- e^{\beta u(r)} \right) \\
&= d(r) \left( 1-e^{\beta u(r)} \right)
        + \left( 1-e^{\beta u(r)} \right) \\
&= \left( d(r) + 1 \right) \left( e^{-\beta u(r)} - 1 \right).
\end{align}
Multiplying both sides by $e^{-\beta u(r)}$ results in the 
Percus-Yevick approximation rewriten as a relation between
the direct and indirect-correlation functions:
\begin{align}
c(r)= f(r)\left( d(r) + 1 \right).
\label{eq:py_c_d}
\end{align}
The Mayer fucntion is finite everywhere, so this last equation can be used in numerical calculations.


The Ornstein-Zernike equation defines the relation between the direct and total-correlation function, but is can be rewriten to relate the direct to the indirect correlation function.
The Fourier transform of the indirect-correlation function is
\begin{align}
D(\kk) &= H(\kk) - C(\kk)\\
&= \frac{C(\kk)}{1 - \rho C(\kk)} - C(\kk),
\label{eq:oz_d_c}
\end{align}
where to I used the Ornstein-Zernike equation $\left( H(\kk) = \frac{C(\kk)}{1 - \rho C(\kk)}\right)$ to go to the second line.


Equations \eqref{eq:py_c_d} and \eqref{eq:oz_d_c} can be used to find a numerical solution for the direct and indirect correlation functions, and the sum of the two gives the total correlation function.

\subsubsection{One More Change to the Naive Algorithm}
The Fourier transformation of a isotropic function in $\mathbb{R}^3$ is (see App.\ref{sec:ft_3d}):

\begin{align}
|\kk|F(|\kk|) = 4 \pi \int_0^\infty dr~|\rr| f(|\rr|) \sin(|\kk||\rr|),
\end{align}
and the inverse transformation is:
\begin{align}
|\rr|f(|\rr|) = \frac{1}{2 \pi^2} \int_0^\infty dk~|\kk| F(|\kk|) \sin(|\kk||\rr|).
\end{align}
These definitions show that it is convenient to work with $|\rr|f(|\rr|)$ and $|\kk|F(|\kk|)$ in stead of $f(|\rr|)$ and $F(|\kk|)$, when one does these transformation repeatedly.
It is therfore expedient to define the following four new functions:
\begin{align}
\tilde{c}(r) &= r c(r), \\
\tilde{d}(r) &= r d(r), \\
\tilde{C}(k) &= k C(k), \\
\tilde{D}(k) &= k D(k),
\end{align}
and use these in the algorithm.
The relations between $\tilde{c}(r)$ and $\tilde{C}(k)$, and
$\tilde{d}(r)$ and $\tilde{D}(k)$ are
\begin{align}\label{eq:c_tilde_ft_pair}
\tilde{C}(k) = 4 \pi \int_0^\infty dr~
    \tilde{c}(r) sin(k r), ~
\tilde{c}(r) = \frac{1}{2 \pi^2} \int_0^\infty dk~
    \tilde{C}(k) sin(k r),
\end{align}
and
\begin{align}\label{eq:d_tilde_ft_pair}
\tilde{D}(k) = 4 \pi \int_0^\infty dr~
    \tilde{d}(r) sin(k r), ~
\tilde{d}(r) = \frac{1}{2 \pi^2} \int_0^\infty dk~
    \tilde{D}(k) sin(k r).
\end{align}
The Ornstein-Zernike equation for the functions $\tilde{C}(k)$ and $\tilde{D}(k)$ is
\begin{align}\label{eq:oz_D_tilde_C_tilde}
\tilde{D}(k) = \frac{\tilde{C}(k)}{1-\rho \tilde{C}(k)/k}
    - \tilde{C}(k).
\end{align}
The division by $k$ in the denominator on the right-hand side
is not a problem because $\tilde{D}(0) = 0$ ($D(k)$ is finite, therfore $kD(k)|_{k=0}=0$), so $k=0$ need not be evaluated.

\subsubsection{The Final Algorithm}
The final version of the algorithm is similar in spirit to the naive one, but uses the transformed functions and the indirect-correlation function in stead of the total-correlation function.
\begin{enumerate}
\item  Guess $\tilde{d}_0(r)$. For example, the low density solution, or the outcome of a previous calculation for different parameters.
\item Get $\tilde{c}_n(r)$ from the modified Percus-Yevick approximation (Eq. \eqref{eq:py_c_d}).
In the first loop through the algorithm $n=0$.
\item Use Eq. \eqref{eq:c_tilde_ft_pair} to get $\tilde{C}_n(k)$.
\item Use Eq. \eqref{eq:oz_D_tilde_C_tilde} to get $\tilde{D}'_{n+1}(k)$:
\begin{align}
\tilde{D}'_{n+1}(k) = \frac{\tilde{C}_n(k)}{1-\rho \tilde{C}_n(k)/k} - \tilde{C}_n(k).
\end{align}
\item Use \eqref{eq:d_tilde_ft_pair} to get $\tilde{d}'_{n+1}(r)$.
\item Mix the new indirect-correlation function with the old one:
\begin{align}
\tilde{d}_{n+1}(r) = \alpha \tilde{d}_n(r) + 
    (1-\alpha) \tilde{d}'_{n+1},
\end{align}
with $0 \leq \alpha < 1$.
\item Check for convergence. If converged, go to 8, else go to 2.
\item  Calculate the total-correlation function from the direct and indirect-correlation functions:
\begin{align}
h(r) = 
\begin{cases}
    -1 ,&\text{if}~~ r = 0\\
    \tilde{c}_n(r)/r + \tilde{d}_n(r)/r ,& \text{if}~~ r>0.
\end{cases}
\end{align}
\end{enumerate}
As a convergence critecion one can use, for example,
\begin{align}
\int dr~\left[ \tilde{d}_n(r) - \tilde{d}'_{n+1}(r) \right]
< \epsilon.
\end{align}
The parameters for the numerical solution are: 1) N, the number of data point for the functions;
2) R, the domain of the functions dependent on space;
(This also defines the domain of the Fourier transformed functions.)
3) $\alpha$, the mixing parameter;
4) $\epsilon$, the convergence parameter.
