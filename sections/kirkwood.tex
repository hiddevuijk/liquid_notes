
The goal of this section is to derive an integral equation for the pair-distribution function called the Kirkwood equation.
For this, one can use a system with two kinds of particles,
just as the system that was used to express the chemical potential using the pair-distribution function. ADD EQ. REF
In this system the interaction potential of particle 1 with any of the other particles is scaled by $\xi$:
\begin{align}
U_N(\rr_1,..,\rr_N) = \sum_{j=2}^{N} \xi u(\rr_1,\rr_j)
+ \sum_{\substack{i=2 \\ j>i}}^N u(\rr_i,\rr_j),
\end{align}
with $0 \leq \xi \leq 1$,
so when $\xi = 1$ it is a system of $N$ identical interacting particles,
and when $\xi = 0$ it is a system consisting of one ideal particle and $N-1$ interacting particles.
Because there are two species of particles in this system (particle 1 and the 
"normal" particles),
this means that one can define two kinds of radial distribution functions:
one for the distribution of the normal particles around particle 1,
one for the distribution of the normale particles around an other normal particle.
In these notes, the difference between these two functions is indicated by the arguments.
When $\rr_1$ is one of the arguments (or $|\rr_1 - \rr_j|$ with $j>2$),
it is the first kind; otherwise it is the second kind.
When which distribution function is which is not clear from the arguments,
I indicate the distribution function of the normal particles around particle 1 with
$g^{(n)}_{(1,n)}(...)$ and the other kind with $g^{(n)}_{(n,n)}(...)$.

The starting point is the identity
\begin{align}
kT~ln\left(\rho^{(n)}(\rr_1,..,\rr_n;\xi)\right) =
&~ kT~ln\left(\rho^{(n)}(\rr_1,..,\rr_n;\xi=0)\right) + \nonumber \\
& \int_0^\xi d\xi' \frac{\partial}{\partial \xi'}
kT~ln\left(\rho^{(n)}(\rr_1,..,\rr_n;\xi')\right).
\end{align}
All that is to be done is to work out what the derivative on the right hand side is.



