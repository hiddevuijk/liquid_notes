
The following definitions of the Fourier transformation in
$\mathbb{R}^3$ are used:
\begin{align}
F(\kk) = \int d \rr e^{-i \kk \cdot \rr} f(\rr),
\end{align}
and
\begin{align}
f(\rr) = \int \frac{d\kk}{\left(2 \pi \right)^3} \ e^{-i \rr \cdot \kk} F(\kk).
\end{align}
When the function $f(\rr)$ only depends on the absolute value
of $\rr$, the forward transformation becomes
\begin{align}
F(\kk) &= \int d \rr e^{-i \kk \cdot \rr} f(|\rr|) \\
&= \int d \rr e^{i k r \cos(\theta)} f(|\rr|),
\end{align}
where the real-space coordinate system is choosen such that
the vector $\kk$ point along the $\hat{e}_z$ axis,
which is possible because of the rotational symmetry of the 
function $f(\rr)$.
The angle $\theta$ is the angle with the $\hat{e}_z$ axis.
This equation also makes it clear that if $f(\rr)$ only depends on the absolute value of $\rr$,
than the Fourier transformation $F(\kk)$ only depends on the absolute value of $\kk$.
Next, one can transform to a spherical coordiante system:
\begin{align}
F(k) &= \int_0^{2 \pi} d\phi \int_0^\pi d \theta \int_0^\infty d r
    r^2  e^{-i r k \cos \left( \theta \right)} f(r) \\
&= - 2 \pi \int_{\cos(0)}^{\cos(\pi)} d\left[ \cos(\theta) \right] \int_0^\infty dr r^2 
    e^{i r k \cos \left( \theta \right)} f(r) \\
&= - 2 \pi \int_{1}^{-1} \int_0^\infty dr r^2
    e^{-i krx} f(r) \\
&= 2 \pi \int_0^\infty dr r^2 f(r) \frac{1}{-ikr}
    \left[ e^{-ikr}-e^{ikr} \right] \\
&= 4 \pi \int_0^\infty dr r^2 f(r) \frac{\sin(kr)}{kr}.
\end{align}
In the same way the inverse transformation can be simpified:
\begin{align}
f(r) &= \int \frac{d\kk}{\left(2 \pi\right)^3} 
F(|\kk|) e^{i \kk \cdot \rr} \\
&= \frac{1}{2 \pi^2} \int_0^\infty dk k^2 F(k)
\frac{\sin(kr)}{kr}.
\end{align}
