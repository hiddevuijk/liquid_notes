
If the integral equation is not linear, it is probably not
possible to solve it analytically and one needs te resort to 
numerical methods.
The integral equations in this course have the form
\begin{align}\label{general_integral_equation}
f(r) = g(r) + \alpha \int dr' K[r,r';f],
\end{align}
where the function $g(r)$ and the kernel $K$ are known
and the goal is to find the function $f(r)$.
The kernel depends on the function $f(r)$ and the parameters $r$ and $r'$;
for example, $K[r,r';f] = g(r')^2 h(r-r')$, with $h$ some other arbitrary function.
Note that in this equation the functions $f(r)$, $g(r)$ and $h(r)$ are arbitrary and are not the Mayer function,
the pair-distribution function and the total-correlation function.
The first step is to guess some acceptable form for the 
function $f(r)$, denoted by $f_0(r)$.
One possibility is
\begin{align}
f_0(r) = \lim_{\alpha \to 0} f(r) = g(r).
\end{align}
The next step is to calculate an imporoved approximation
to the function $f(r)$ by evaluating the right hand side of
Eq. \eqref{general_integral_equation} with $f_0(r)$.
Repeating this last step until the new approximation does not
differ from the previous one, gives the final solution.
So, the improved solution is
\begin{align}
f_{n+1}(r) = g(r) + \alpha \int dr' K[r,r';f_n],
\end{align}
and the final solution is obtained when $f_{n+1}(r) \approx f_n(r)$.
\footnote{Of course, for the numerical evaluation of the integrals, they have to be approximated by a sum.
For numerical algorithms for integration, see for example
NUMERICAL RECIPES.}
The convergence of this method can be improved by mixing the old with the new solution:
\begin{align}
f_{n+1}(r) = \eta f'_{n+1}(r) + (1-\eta) f_n(r),
\end{align}
with $0 < \eta \leq 1$ and 
\begin{align}
f'_{n+1}(r) = g(r) + \alpha \int dr' K[r,r';f_n].
\end{align}
As a convergence criteria one can use, for example,
\begin{align}
\int dr \left( f'_{n+1}(r) - f_n(r) \right)^2 < \epsilon,
\end{align}
for some $\epsilon$.


