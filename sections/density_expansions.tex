\subsubsection{Distribution Functions}
The density expansions of distribution functions can be done in a similar way as the virial expansion.
We work in the grand canonical ensemble, and start with the 
definition of the two-body distribution function:
\begin{align}
\rho^{(2)} (\rr_1,\rr_2) &= \frac{1}{\Xi} \sum_{N=2}^\infty
    \frac{z^N}{(N-2)!} \int d\rr_3 \cdots \int d\rr_N
    e^{-\beta U_N(\rr^N)}\\
&= \frac{1}{\Xi} \left[ z^2 e^{-\beta u(r_{1,2})} +
  \sum_{N=2}^\infty \frac{z^N}{(N-2)!} \int d\rr_3 \cdots \int d\rr_N
    e^{-\beta U_N(\rr^N)} \right],\\
&=  \frac{z^2 e^{-\beta u(r_{1,2})}}{\Xi} \left[ 1 +
  \sum_{N=2}^\infty \frac{z^{N-2}}{(N-2)!} \int d\rr_3 \cdots \int d\rr_N
    e^{-\beta U'_N(\rr^N)} \right],
\label{eq:rho2_exp1}
\end{align}
where I assumed a pairwise additive interaction potential, and defined $U'_N(\rr_3,..,\rr_N) = -u(\rr_{1,2}) + \sum_{i<j}^N
u(\rr_{i,j})$.
Note that the activity, $z$, can be expanded in the density (see REF) $z = \rho -2b_2 \rho^2 +(8b_2^2 - 3b_3) \rho^3 + ...$.
For a density expansion up to $\mathcal{O}(\rho^n)$,
one needs to consider terms up to $\mathcal{O}(z^n)$.
So, the term in the square bracktets already has a form of a density expansion.
One only has to expand the inverse of the partition function in the activity, put in the density expansion of the activity and reorder the result in powers of density.

From the discussion on the virial expansion  we know that
\begin{align}
\Xi = 1 + \sum_{N=1}^{\infty} \frac{z^N}{N!} Z_N,
\end{align}
where $Z_N$ is the configurational integral.
This also gives the expansion of the inverse of the partition function:
\begin{align}
\frac{1}{\Xi} &= 1 - \sum_{N=1}^{\infty} \frac{z^N}{N!} Z_N +
    \left( \sum_{N=1}^{\infty} \frac{z^N}{N!} Z_N \right)^2 + \cdots \\
&= 1 - \left( z Z_1 + \frac{1}{2} z^2 Z_2 + \cdots \right) +
    \left(z Z_1 + \frac{1}{2} z^2 Z_2 + \cdots \right)^2 + \cdots \\
&= 1 - zZ_1  - \frac{1}{2} z^2 Z_2 + z^2 Z_1^2 + \mathcal{O}(z^3) \\
&= 1 - z Z_1 - \frac{1}{2} z^2 \left( Z_2 - 2 Z_1^2 \right) +
    \mathcal{O}(z^3).
\label{eq:xi_exp}
\end{align}

Using this expression for the inverse of the partition function in Eq. \eqref{eq:rho2_exp1} gives
\begin{align}
\rho^{(2)}(\rr_1,\rr_2) &= z^2 e^{-\beta u(\rr_{1,2})}
    \underbrace{
    \left[ 1 - z Z_1 \right]
    }_{1/\Xi}
    \underbrace{
    \left[ 1 + z \int d\rr_3 e^{-\beta U'_3(\rr_1,\rr_2,\rr_3)} \right]}_\text{square bracktet in Eq. \eqref{eq:rho2_exp1}}
    + \mathcal{O}(z^4) \\
&= z^2 e^{-\beta u(\rr_{1,2})} \left\{1 - z \left[ Z_1 - \int d\rr_3 e^{-\beta U'_3(\rr_1,\rr_2,\rr_3)} \right] \right\} + \mathcal{O}(z^4).
\label{eq:rho2_exp2}
\end{align}
The expansion of $z$ is
$z = \rho - 2b_2\rho^2 + \mathcal{O}(\rho^3)$
with $b_2 = \frac{1}{2 V} (Z_2-Z_1^2)$ (see REF).
The expansion of $z^2$ is therefor
$z^2 = \rho^2 - 4 b_2 \rho^3 + \mathcal{O}(\rho^4)$.
Replacing $z$ and $z^2$ in Eq. \eqref{eq:rho2_exp2} gives
\begin{align}
\rho^{(2)}(\rr_1,\rr_2) &=
    \underbrace{ \left[ \rho^2 - 4b_2 \rho^3 \right] }_{z^2}
     e^{-\beta u(\rr_{1,2})} \left\{1 - 
    \underbrace{\rho}_{z} \left[ Z_1 - \int d\rr_3 e^{-\beta U'_3(\rr_1,\rr_2,\rr_3)} \right] \right\} + \mathcal{O}(z^4)\\
&= e^{-\beta u(\rr_{1,2})} \left[\rho^2 - 
    \rho^3 \left(4b_2 + Z_1 - \int d\rr_3 e^{-\beta U'_3(\rr_1,\rr_2,\rr_3)} \right) \right] + \mathcal{O}(z^4).
\label{eq:rho2_exp3}
\end{align}
Next we need to find an expression for the coefficient of the $\rho^3$ term.
They were already evaluated when we discussed the virial expansion, but I will show it here again.
For $4 b_2 = \frac{2}{V} \left( Z_2 - Z_1^2 \right)$ (See REF)
we need to know $Z_1$ and $Z_2$.
First $Z_2$, this is, by definition,
\begin{align}
\frac{2}{V} Z_2 &= \frac{2}{V} \int d\rr'_1 \int d\rr'_2
    e^{-\beta u(|\rr'_1 - \rr'_2|)} \\
&= 2 \int d\rr_3 e^{-\beta u(|\rr'_1 - \rr_3|)}\\
&= 2 \int d\rr_3 e^{-\beta u(|\rr'_2 - \rr_3|)}.
\end{align}
To go from the first to the second line, I changed coordinates from $d\rr'_2$ to  $d\rr_3$ with
$d\rr_3 = d\rr'_1 - d\rr'_2$, than the integrand is 
independent of $\rr'_1$.
So the $\rr'_1$ integral gives a factor of $V$.
Then shifting the origin of $\rr_3$ to $\rr'_1$ gives the 
second line, and shifting the origin to $\rr'_2$ gives the 
third line.
Shifting the origin is allowed because the system is homogeneous.
Adding half of the second line to half of the third line gives
\begin{align}
\frac{2}{V}Z_2 = \int d\rr_3 
e^{-\beta u(|\rr_1 - \rr_3|)} + 
e^{-\beta u(|\rr_2 - \rr_3|)}.
\end{align}
The configuration integral for one particle is equal to the volume which can be writen as $\int d\rr_3$, so
\begin{align}
\frac{2}{V} Z_1^2 &= 2 V \\
&= 2 \int d\rr_3.
\end{align}
With the expression for $\frac{2}{V}Z_2$ and $\frac{2}{V}Z_1^2$ we have
\begin{align}
\frac{2}{V} b_2 &= \int d\rr_3 
    e^{-\beta u(|\rr_1 - \rr_3|)} + 
    e^{-\beta u(|\rr_2 - \rr_3|)} - 2\\
&= \int d\rr_3 \left\{ f_{13} + f_{23} \right\},
\end{align}
where in the last line I used the definition of the Mayer functions $f_{13} = e^{-\beta u(|\rr_1 - \rr_3|)} - 1$,
and similarly for $f_{23}$.

We can do the same kind of manipulations to the other part
of the coefficient of $\rho^3$ in Eq. \eqref{eq:rho2_exp3}:
\begin{align}
Z_1 -  \int d\rr_3 e^{-\beta U'_3(\rr_1,\rr_2,\rr_3)} &=
    - \int d\rr_3 e^{-\beta u(|\rr_1 - \rr_3|)} 
    e^{-\beta u(|\rr_1 - \rr_3|)} -1 \\
&= - \int d\rr_3 \left(f_{13}+ 1\right)\left(f_{23}+ 1\right) -1\\
&= - \int d\rr_3 f_{13}+ f_{23} + f_{13}f_{23},
\end{align}
where I used the definition of $U'_3$ and used $Z_1=V=\int d\rr_3$.

The coefficient of the $\rho^3$ term in Eq. \eqref{eq:rho2_exp3} can now be expressed in terms of Mayer functions, and the density expansion can be writen as
\begin{align}
\rho^{(2)}(\rr_1,\rr_2) = e^{-\beta u(r_{12})}
    \left[ \rho^2 + \rho^3 \int d\rr_3 f_{13}f_{23} \right]
    + \mathcal{O}(\rho^4).
\end{align}
One could, with considerably more effort, calculated the density expansion one order higher.
This would result in
\begin{align}
\rho^{(2)}(\rr_1,\rr_2) = e^{-\beta u(r_{12})}
    \Bigg[& \rho^2 + \rho^3 \int d\rr_3 f_{13}f_{23} 
        +\frac{1}{2} \rho^4 \int d\rr_3 \int d\rr_4
        \big( 2 f_{13}f_{34}f_{24} + \nonumber \\
            &4f_{13}f_{34}f_{23}f_{24} +
            f_{13}f_{23}f_{14}f_{24} +
            f_{13}f_{23}f_{14}f_{24}f_{34}
        \big)
    \Bigg]
    + \mathcal{O}(\rho^5).
\end{align}

\subsubsection{Correlation Functions}
