

The goal of this section is to derive an integral equation for the pair-distribution function called the Kirkwood equation.
The derivation shown here is based on Refs. \cite{hill1987statistical} (section 32) and \cite{mcquarrie2000statistical} (chapter 13-4).



For this, one can use a system with two kinds of particles,
just as the system that was used to express the chemical potential using the pair-distribution function. ADD EQ. REF
In this system the interaction potential of particle 1 with any of the other particles is scaled by $\xi$:
\begin{align}\label{eq:kirkwood:UN_xi}
U_N(\rr_1,..,\rr_N) = \sum_{j=2}^{N} \xi u(\rr_1,\rr_j)
+ \sum_{\substack{i=2 \\ j>i}}^N u(\rr_i,\rr_j),
\end{align}
with $0 \leq \xi \leq 1$,
so when $\xi = 1$ it is a system of $N$ identical interacting particles,
and when $\xi = 0$ it is a system consisting of one ideal particle and $N-1$ interacting particles.
Because there are two species of particles in this system (particle 1 and the 
"normal" particles),
this means that one can define two kinds of radial distribution functions:
one for the distribution of the normal particles around particle 1,
one for the distribution of the normale particles around an other normal particle.
In these notes, the difference between these two functions is indicated by the arguments.
When $\rr_1$ is one of the arguments (or $|\rr_1 - \rr_j|$ with $j>2$),
it is the first kind; otherwise it is the second kind.
When which distribution function is which is not clear from the arguments,
I indicate the distribution function of the normal particles around particle 1 with
$g^{(n)}_{(1,n)}(...)$ and the other kind with $g^{(n)}_{(n,n)}(...)$.

The starting point is the identity
\begin{align}\label{eq:kirkwood:start}
kT~ln\left(\rho^{(n)}(\rr_1,..,\rr_n;\xi)\right) =
&~ kT~ln\left(\rho^{(n)}(\rr_1,..,\rr_n;\xi=0)\right) + \nonumber \\
& \int_0^\xi d\xi' \frac{\partial}{\partial \xi'}
kT~ln\left(\rho^{(n)}(\rr_1,..,\rr_n;\xi')\right).
\end{align}
First we calculate
\begin{align}
kT\frac{\partial}{\partial \xi}
    \rho^{(n)}(\rr_1,\cdots,\rr_n;\xi) &=
kT\left( \frac{\partial}{\partial \xi}\frac{1}{Z_N(\xi)} \right)
    \frac{N!}{(N-n)!}
    \int d\rr_{n+1} \cdots \int d\rr_N e^{-\beta U_N(\xi)}
        \nonumber \\
&~~+ kT\frac{1}{Z_N(\xi)} \frac{N!}{(N-n)!}
    \int d\rr_{n+1} \cdots \int d\rr_N 
    \frac{\partial}{\partial \xi} e^{-\beta U_N(\xi)}
        \nonumber \\
&= -kT\rho^{(n)}(\rr_1,\cdots,\rr_n;\xi)
        \frac{1}{Z_N(\xi)} \frac{\partial}{\partial \xi}
            Z_N(\xi)
        \nonumber \\
    &~~-\frac{N!}{(N-n)!} \frac{1}{Z_N(\xi)}
        \int d\rr_{n+1} \cdots \int \rr_N e^{-\beta U_N}
        \sum_{j=2}^{N} u(\rr_1,\rr_j).
    \nonumber 
\end{align}
The terms in the sum with $j<n+1$ can be taken out of the integral:
\begin{align} \label{eq:kirkwood:int1}
kT\frac{\partial}{\partial \xi}
    \rho^{(n)}(\rr_1,\cdots,\rr_n;\xi)
&= -kT\rho^{(n)}(\rr_1,\cdots,\rr_n;\xi)
        \frac{1}{Z_N(\xi)} \frac{\partial}{\partial \xi}
        Z_N(\xi) \nonumber \\
    &~~- \sum_{j=2}^{n} u(\rr_1,\rr_j)
        \underbrace{
        \frac{N!}{(N-n)!} \frac{1}{Z_N(\xi)}
        \int d\rr_{n+1} \cdots \int \rr_N e^{-\beta U_N}
        }_{= \rho^{(n)}(\rr_1,\cdots,\rr_n;\xi)}
        \nonumber \\
    &~~-\frac{N!}{(N-n)!} \frac{1}{Z_N(\xi)}
        \int d\rr_{n+1} \cdots \int \rr_N e^{-\beta U_N}
        \sum_{j=n+1}^{N} u(\rr_1,\rr_j).
\end{align}
Each term in the last sum in the contribute the same.
So you can pick one, say $j=n+1$, and multiply by the number
of terms in the sum, which is $N-n$.
And we also need
\begin{align}
-kT\frac{1}{Z_N(\xi)} \frac{\partial}{\partial \xi} Z_N(\xi)
= \frac{1}{Z_N(\xi)}
    \int \rr_1 \cdots \int \rr_N e^{-\beta U_N(\xi)}
       \sum_{j=2}^N u(\rr_1,\rr_j). 
\nonumber
\end{align}
And again each term in the sum contributes the same,
so you can pick the term with $j=2$ and multiply by $N-1$.
So Eq. \eqref{eq:kirkwood:int1} becomes
\begin{align}
kT\frac{\partial}{\partial \xi}
    \rho^{(n)}(\rr_1,\cdots,\rr_n;\xi)
&= \rho^{(n)}(\rr_1,\cdots,\rr_n;\xi)
        \int d\rr_1' \int d\rr_2' u(\rr_1',\rr_2')
        \underbrace{
        \frac{N-1}{Z_N(\xi)} 
        \int d\rr_3 \cdots \int d\rr_N e^{-\beta U_N} 
        }_{=\frac{1}{N}\rho^{(1)}(\rr_1',\rr_2';\xi)}
         \nonumber \\
    &~~- \rho^{(n)}(\rr_1,\cdots,\rr_n;\xi)
        \sum_{j=2}^{n} u(\rr_1,\rr_j)
        \nonumber \\
    &~~- \int d\rr_{n+1}' u(\rr_1,\rr_{n+1}')
        \underbrace{
        \frac{N!}{(N-n)!} \frac{1}{Z_N(\xi)}
        \int d\rr_{n+2}'\cdots \int \rr_N e^{-\beta U_N}
        }_{=\rho^{(n+1)}(\rr_1,\cdots,\rr_{n+1}';\xi)}.
\nonumber
\end{align}
For Eq. \eqref{eq:kirkwood:start} we need the derivative of the $ln$ of $\rho^{(n)}$, which is the same as  the previous equation devided by$\rho^{(n)}$.
\begin{align}
kT\frac{\partial}{\partial \xi}
    ln\left(\rho^{(n)}(\rr_1,\cdots,\rr_n;\xi) \right)
=&- \sum_{j=2}^{n} u(\rr_1,\rr_j)
    - \int d\rr_{n+1}' u(\rr_1,\rr_{n+1}')
        \frac{ \rho^{(n+1)}(\rr_1,\cdots,\rr_{n+1}';\xi)}
            { \rho^{(n)}(\rr_1,\cdots,\rr_{n};\xi)}
    \nonumber \\
&~- \frac{1}{N} \int d\rr_1' \int d\rr_2' u(\rr_1',\rr_2')
        \rho^{(1)}(\rr_1',\rr_2';\xi).
\nonumber
\end{align}

\subsubsection{interpretation with potential of mean force}

OLD STUFF FOR N=2

All that is to be done is to work out what the derivative on the right hand side is.
For simplicity, I will do the $n=2$ case here, and show the general case in App. .
We need the derivative with respect of $\xi$ of
\begin{align}
\rho^{(2)}(\rr_1,\rr_2;\xi) =
\frac{N(N-1)}{Z_N(\xi)} \int d\rr_3 \cdots \int d\rr_N~
e^{-\beta U_N(\xi)}.
\end{align}
This is
\begin{align}\label{eq:kirkwood:first_deriv}
k T \frac{\rho^{(2)}(\partial \rr_1,\rr_2;\xi)}{\partial \xi} &=
\frac{N(N-1)}{Z_N(\xi)} \int d\rr_3 \cdots \int d\rr_N~
    e^{-\beta U_N(\xi)}. \\
&= 
    \underbrace{kT N (N-1) \left( \frac{\partial}{\partial \xi}
    \frac{1}{Z_N(\xi)} \right)
    \int d\rr_3 \cdots \int d\rr_N~ e^{-\beta U_N(\xi)}
    }_{ \equiv I_1}
    \nonumber \\
    &~~~~~~
    +\underbrace{ kT \frac{N(N-1)}{Z_N(\xi)}
    \int d\rr_3 \cdots \int d\rr_N~
    \frac{\partial}{\partial \xi}e^{-\beta U_N(\xi)}
    }_{\equiv I_2}.
\end{align}
The first integral is
\begin{align}
I_1 &= -\frac{kT N(N-1)}{Z_n(\xi)^2}
    \left( \frac{\partial}{\partial \xi}
        \underbrace{
        \int d\rr_1 \cdots \int d\rr_N~ e^{-\beta U_N(\xi)}
        }_{Z_N(\xi)}
    \right)
    \int d\rr_3 \cdots \int d\rr_N~ e^{-\beta U_N(\xi)} \\
&= \rho^{(2)}(\rr_1,\rr_2;\xi) \frac{-kT}{Z_N(\xi)}
    \int d\rr_1 \cdots \int d\rr_N~ e^{-\beta U_N(\xi)}
        \frac{\partial}{\partial \xi}
        \left( - \beta U_N(\xi) \right).
\end{align}
The $-\beta$ cancels the $-kT$ in the fraction, and the derivative in the integral is $\partial_\xi U_N = \sum_{j=2}^N u(\rr_1,\rr_j)$ (see \eqref{eq:kirkwood:UN_xi}).
There are $N-1$ terms in this sum.
Each term has the same contribution because the $\rr$'s are all integration parameters.
Therefore, one can pick one of the terms in the sum, say $u(\rr_1,\rr_2)$, and multiply the result by $N-1$.
This results in
\begin{align}
I_1 &= \rho^{(2)}(\rr_1,\rr_2;\xi)\frac{1}{N}
    \int d\rr_1 \int d\rr_2 u(\rr_1,\rr_2)
    \frac{N(N-1)}{Z_N(\xi)}
    \int d\rr_3 \cdots \int d\rr_N~ e^{-\beta U_N(\xi)},
\end{align}
where I multiplied and divided by N because the second fraction and the second set of integrals is $\rho^{(2)}(\rr_1,\rr_2;\xi)$.
So the integral is
\begin{align}
I_1 &= \rho^{(2)}(\rr_1,\rr_2;\xi)\frac{1}{N}
    \int d\rr_1 \int d\rr_2 u(\rr_1,\rr_2)
    \rho^{(2)}(\rr_1,\rr_2;\xi).
\end{align}

For the second integral in Eq. \eqref{eq:kirkwood:first_deriv} we need
\begin{align}
kT \frac{\partial}{\partial \xi} 
    e^{-\beta U_N(\xi)} 
&= - e^{-\beta U_N(\xi)} \frac{\partial}{\partial \xi}
    U_N \\
&= - e^{-\beta U_N(\xi)} \sum_{j=2}^N u(\rr_1,\rr_j) \\
&=- e^{-\beta U_N(\xi)}\left(
    u(\rr_1,\rr_2) +  \sum_{j=3}^N u(\rr_1,\rr_j)
    \right).
\end{align}
The sum in the last line has $N-2$ terms, so one can
replace the sum by a single term,say $u(\rr_1,\rr_3)$, and multiply by the number of terms.
This gives
\begin{align}
I_2 &= -u(\rr_1,\rr_2) \frac{N(N-1)}{Z_N(\xi)}
    \int d\rr_3 \cdots \int d\rr_N~ e^{-\beta U_N(\xi)}\\
    &~~~~~- \frac{N(N-1)(N-2)}{Z_N(\xi)}
        \int d\rr_3 u(\rr_1,\rr_3) \int d\rr_4 \cdots
        \int d\rr_N~ e^{-\beta U_N(\xi)} \\
&= -u(\rr_1,\rr_2)\rho^{(2)}(\rr_1,\rr_2;\xi) 
    - \int d\rr_3 u(\rr_1,\rr_3) \rho^{(3)}(\rr_1,\rr_2,\rr_3;\xi).
\end{align}

From Eq. \eqref{eq:kirkwood:start} with $n=2$ we have
\begin{align}
kT~ln\left(\rho^{(n)}(\rr_1,..,\rr_n;\xi)\right) &=
     kT~ln\left(\rho^{(n)}(\rr_1,..,\rr_n;\xi=0)\right) + \nonumber \\
    &~~~~~~~~ \int_0^\xi d\xi' \frac{\partial}{\partial \xi'}
    kT~ln\left(\rho^{(n)}(\rr_1,..,\rr_n;\xi')\right) \\
&= kT ln(\rho^2) +
    \int_0^\xi d\xi'
    \frac{1}{\rho^{(2)}(\rr_1,\rr_2;\xi')} kT
        \frac{\partial}{\partial \xi}
    \rho^{(2)}(\rr_1,\rr_2;\xi') \\
&= kT ln(\rho^2) +
    \int_0^\xi d\xi'
    \frac{1}{\rho^{(2)}(\rr_1,\rr_2;\xi')} \left(I_1+I_2 \right),
\end{align}
where to go from the first to the second line I used that $\rho^{(2)}(\rr_1,\rr_2;\xi=0)=\rho^2 g^{(2)}(\rr_1,\rr_2;\xi=0) = \rho^2$ because if particle 1 does not interact with any of the other particles it is not correlated with any other particle.
Replacing $I_1$ and $I_2$ with their explicit expressions
results in
\begin{align}
kT~ln\left(\rho^{(n)}(\rr_1,..,\rr_n;\xi)\right)
&=
    kT ln(\rho^2) + \int_0^\xi d\xi' \Bigg( \nonumber \\
&    \frac{1}{N} \int d\rr_1' \int d\rr_2' u(\rr_1',\rr_2')
        \rho^{(2)}(\rr_1',\rr_2';\xi')
\nonumber \\
& -u(\rr_1,\rr_2)  - \int d\rr_3 u(\rr_1,\rr_3)\frac{ \rho^{(3)}(\rr_1,\rr_2,\rr_3;\xi)}{\rho^{(2)}(\rr_1,\rr_2;\xi)} \Bigg) \\
&= kT ln(\rho^2) - \xi u(\rr_1,\rr_2) + \nonumber \\
   &~~~~ \frac{1}{N} \int d\rr_1' \int d\rr_2' \int_0^\xi d\xi' u(\rr_1',\rr_2') \rho^{(2)}(\rr_1',\rr_2';\xi')
    -  \nonumber \\
&~~~~ \int_0^\xi d\xi' u(\rr_1,\rr_3) 
    \frac{\rho^{(3)}(\rr_1,\rr_2,\rr_3;\xi')}
       {\rho^{(2)}(\rr_1,\rr_2;\xi')}
\end{align}




